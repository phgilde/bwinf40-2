\documentclass[a4paper,10pt,ngerman]{scrartcl}
\usepackage{babel}
\usepackage{hyperref}
\usepackage[T1]{fontenc}
\usepackage[utf8x]{inputenc}
\usepackage[a4paper,margin=2.5cm,footskip=0.5cm]{geometry}

\newcommand{\Aufgabe}{Aufgabe 4: Zara Zackigs Zurückkehr} % Aufgabennummer und Aufgabennamen angeben
\newcommand{\TeamId}{62454}                       % Team-ID aus dem PMS angeben
\newcommand{\TeamName}{SilverBean}                 % Team-Namen angeben
\newcommand{\Namen}{Philip Gilde}           % Namen der Bearbeiter/-innen dieser Aufgabe angeben
 \newcommand*\xor{\oplus}
\usepackage{scrlayer-scrpage, lastpage}
\setkomafont{pageheadfoot}{\large\textrm}
\lohead{\Aufgabe}
\rohead{Teilnahme-ID: \TeamId}
\cfoot*{\thepage{}/\pageref{LastPage}}

% Position des Titels
\usepackage{titling}
\setlength{\droptitle}{-1.0cm}

% Für mathematische Befehle und Symbole
\usepackage{amsmath}
\usepackage{mathtools}
\DeclareMathOperator*{\argmax}{arg\,max}
\usepackage{amssymb}

% Für Bilder
\usepackage{graphicx}
\graphicspath{ {./bilder/}}
% Für Algorithmen
\usepackage{algpseudocode}

% Für Quelltext
\usepackage{listings}
\lstset{literate=%
  {Ö}{{\"O}}1
  {Ä}{{\"A}}1
  {Ü}{{\"U}}1
  {ß}{{\ss}}1
  {ü}{{\"u}}1
  {ä}{{\"a}}1
  {ö}{{\"o}}1
}
\usepackage{wrapfig}
\usepackage{color}
\definecolor{mygreen}{rgb}{0,0.6,0}
\definecolor{mygray}{rgb}{0.5,0.5,0.5}
\definecolor{mymauve}{rgb}{0.58,0,0.82}
\lstset{
  keywordstyle=\color{blue},commentstyle=\color{mygreen},
  stringstyle=\color{mymauve},rulecolor=\color{black},
  basicstyle=\footnotesize\ttfamily,numberstyle=\tiny\color{mygray},
  captionpos=b, % sets the caption-position to bottom
  keepspaces=true, % keeps spaces in text
  numbers=left, numbersep=5pt, showspaces=false,showstringspaces=true,
  showtabs=false, stepnumber=2, tabsize=2, title=\lstname
}
\lstdefinelanguage{JavaScript}{ % JavaScript ist als einzige Sprache noch nicht vordefiniert
  keywords={break, case, catch, continue, debugger, default, delete, do, else, finally, for, function, if, in, instanceof, new, return, switch, this, throw, try, typeof, var, void, while, with},
  morecomment=[l]{//},
  morecomment=[s]{/*}{*/},
  morestring=[b]',
  morestring=[b]",
  sensitive=true
}
\usepackage{etoolbox}
\patchcmd{\thebibliography}{\chapter*}{\section}{}{}
\usepackage{csquotes}
\newcommand\mdoubleplus{\mathbin{+\mkern-10mu+}}
% Diese beiden Pakete müssen zuletzt geladen werden
%\usepackage{hyperref} % Anklickbare Links im Dokument
\usepackage{cleveref}

% Daten für die Titelseite
\title{\textbf{\Huge\Aufgabe}}
\author{\LARGE Teilnahme-ID: \LARGE \TeamId \\\\
	    \LARGE Bearbeiter dieser Aufgabe: \\ 
	    \LARGE \Namen\\\\}
\date{\LARGE\today}

\begin{document}

\maketitle
\tableofcontents

\vspace{0.5cm}



\section{Lösungsidee}
Von den gegebenen $n$ Karten mit jeweils $m$ Bits werden $p$ Karten gesucht, so dass das exklusive Oder (im Folgenden als XOR abgekürzt) von $p-1$ Karten gleich der $p.$ Karte ist.
\begin{align*}
&& k_1 \xor k_2 \xor \ldots \xor k_{p-1} &= k_p && |\xor k_p \\
\leftrightarrow && k_1 \xor k_2 \xor \ldots \xor k_{p-1} \xor k_p &= 0 &&
\end{align*}
Diese Gleichung lässt sich zu jeder der $p$ Karten umstellen. Es werden also $p$ Karten gesucht, deren XOR gleich einer Karte mit $m$ Nullen ist. Dieses Problem lässt sich umformulieren zu einem linearen Gleichungssystem im Galois-Feld $GF(2)$. Dieses besteht nur aus den beiden Elementen 0 und 1. Die Addition im Feld entspricht dem XOR, die Multiplikation einem UND. Die gemischten Karten entsprechen der Matrix $K \in GF(2)^{n \times m}$. $K_n$ ist dabei die n-te Karte und $K_{n,m}$ das m-te Bit der n-ten Karte. Gesucht wird der Vektor $v \in GF(2)^n$, so dass dieses lineare Gleichungssystem gilt:
\begin{align*}
K_{1, 1} v_1 + K_{2, 1} v_2 + \ldots + K_{n, 1} v_n &= 0 \\
K_{1, 2} v_1 + K_{2, 2} v_2 + \ldots + K_{n, 2} v_n &= 0 \\
\ldots \\
K_{1, m} v_1 + K_{2, m} v_2 + \ldots + K_{n, m} v_n &= 0 \\ 
\end{align*}
In Matrixform:
\begin{align*}
K^T v = 0
\end{align*}
Dabei bestimmt $v_n$, ob die n-te Karte zu den gesuchten Karten gehört. \\
Die Menge von Vektoren, die sich oben für $v$ einsetzen lassen, wird als Nullraum oder Kern der Matrix $K^T$ bezeichnet. In $GF(2)$ kann ein Vektor nicht skaliert werden, weil er nur mit entweder $0$ oder $1$ multipliziert werden kann. Somit besteht der Nullraum aus allen möglichen Kombinationen von Summen der Basisvektoren. Wenn $r$ der Rang von $K^T$ ist, dann ist $q=n-r$ die Anzahl der Basisvektoren des Nullraums. Der Rang von $K^T$ ist die Anzahl linear unabhängiger Zeilen beziehungsweise Spalten (diese beiden Werte sind gleich). Wenn, wie in der ursprünglichen Aufgabe, $n<m$, dann ist der Rang in der Regel $n-1$, denn nur eine Karte, die Wiederherstellungskarte, ist linear abhängig von den anderen. \\
Die Wahrscheinlichkeit, dass $h$ Karten mit jeweils $b$ Bits voneinander linear unabhängig sind, ist, wenn diese zufällig und erzeugt sind und Nullen und Einsen gleich wahrscheinlich sind, wovon der Einfachheit halber ausgegangen wird, nach \cite{WEBSITE:1} gegeben durch
\begin{align*}
p_h = \prod^h_{i=1} (1-2^{i-1-b})
\end{align*}
Diese ist für $h=n-1=110$ und $b=m=128$ hoch genug, um den anderen Fall zunächst außen vor zu lassen. Somit sind alle bis auf eine der $111$ Karten linear unabhängig voneinander. Der Nullraum besteht damit aus nur $q=n-(n-1)=n-n+1=1$ Vektor. Dieser muss an den Stellen der $11$ echten Karten $1$ und sonst überall $0$ sein. Somit haben wir die echten Karten gefunden. \\
Um die Basisvektoren zu finden, wird das in \cite{WEBSITE:2} beschriebene Verfahren verwendet. Dabei wird zuerst $K^T$ mit der Identitätsmatrix zu $\left[\frac{K^T}{I}\right]$ erweitert. $\left[\frac{K^T}{I}\right]^T$ wird dann mithilfe des Gauß-Jordan-Algorithmus in Stufenform gebracht, was dann in der Matrix $\left[\frac{B}{C}\right]^T$ resultiert. Die Basis des Nullraums bilden die Spalten von $C$, deren entsprechende Spalten in $B$ Null sind. Das lässt sich damit begründen, dass die elementaren Reihentransformationen der Transponierten beziehungsweise elementaren Spaltentransformationen der Multiplikation mit einer Matrix $P$ entsprechen, also $\left[\frac{K^T}{I}\right] P = \left[\frac{B}{C}\right]$. Daraus folgt $I P = P = C$ und $K^T P = K^T C = B$.
\begin{align*}
&& K^T C &= B  &&|\cdot C^{-1}\\
\leftrightarrow && K^T &= BC^{-1} &&|\cdot v \\
\leftrightarrow && K^T v &= BC^{-1} v && | \text{Es wird $C^{-1}v = w$ gesetzt}\\
&& &= Bw && | \text{Damit $v$ zum Nullraum gehört, wird verlangt:}\\
&& &= 0 
\end{align*}
Weil alle Spalten in $B$, die nicht Null sind, linear unabhängig voneinander sind (Das bedeutet nämlich Spaltenstufenform), gilt $Bw=0$ nur wenn die Einträge von $w$, die nicht Null sind, den Nullspalten von $B$ entsprechen. Die Basis der Vektoren $w$, für die $Bw=0$ gilt, sind also die verschiedenen Vektoren, die eine Eins bei einer Nullspalte von $B$ und sonst überall Nullen haben. Da $C^{-1} v = w \leftrightarrow v = C w$ definiert wurde, sind die Spalten von $C$, die den Nullspalten von $B$ entsprechen, die Basis des Nullraums. \\\\
Wenn $q>1$ ist, kann es sein, dass keiner der Basisvektoren des Nullraums 11 Einsen beinhaltet, sondern eine Linearkombination dieser. In diesem Fall wird jede der $2^q$ Kombinationen der Basisvektoren durchprobiert, bis eine davon $11$ Eisen enthält.\\\\
Die richtige Karte für das $s$-te Haus kann gefunden werden, in dem man die Karten aufsteigend sortiert und zuerst die $s$-te und dann die $s+1$-te Karte ausprobiert. Wenn die Sicherungskarte kleiner als die Öffnungskarte des $s$-ten Hauses ist, liegt sie davor im Stapel und die Öffnungskarte an der Stelle $s+1$. Wenn sie größer ist, dann liegt sie dahinter im Stapel und die Öffnungskarte an Stelle $s$.\\
\\
Das Verfahren stößt an seine Grenzen, wenn $n>m$ ist. Der Rang von $K^T$ ist dann nämlich höchstens $m$, wodurch der Nullraum $q=n-m$ Basisvektoren hat. Diese Basisvektoren müssen nun nicht zwangsweise einen mit 11 Einsen beinhalten, dieser könnte auch eine Linearkombination der anderen Basisvektoren sein. Wenn das der Fall ist, müsste man alle $2^q$ Kombinationen von Basisvektoren durchprobieren, bis eine davon $11$ Einsen beinhaltet, was für die Beispieleingabe auf der BwInf-Webseite mit $161$ Karten zu je $128$ Bit $2^q=2^{n-m}=2^{161-128}=2^{33}=8.589.934.592$ Kombinationen sind. Diese lassen sich nicht wirklich in überschaubarer Zeit durchprobieren.
\section{Umsetzung}
Die Lösung wurde in Python umgesetzt. Dabei wurde die Bibliothek \lstinline|NumPy| verwendet, um die Koeffizientenmatrizen darzustellen. 
\section{Laufzeit}
Ein Durchlauf des Gauß-Jordan-Algorithmus hat eine Laufzeit von $\mathcal{O}(n^3)$. Der Algorithmus wird einmal ursprünglich und dann für jeden der $q$ Basisvektoren des Nullraums durchgeführt, somit ist die Laufzeit $\mathcal{O}((q+1)n^3)$
\section{Beispiele}

\section{Quellcode}

\bibliography{referenzen}
\bibliographystyle{ieeetr}
\end{document}
